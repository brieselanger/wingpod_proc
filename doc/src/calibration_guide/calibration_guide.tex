\documentclass[a4paper]{article} % legt den Dokumenttyp fest\\
\usepackage[utf8x]{inputenc} % legt fest, welche Sprach- und Zeichenpakete zur Darstellung des Textes verwendet werden sollen
\usepackage[ngerman]{babel}
\usepackage{amssymb} % Paket für mathematische Formeln
\usepackage{amsmath}
\usepackage{bm} % big math mode
\usepackage[margin=10pt,font=small,labelfont=bf,textfont=it]{caption} %Caption-Formatierung
\usepackage{wrapfig} % Text um Bild fließen lassen
\usepackage{booktabs} % wissenschaftliche Tabellensätze
\usepackage{geometry} % Seitenränder anpassen
\usepackage{textcomp}
\geometry{a4paper,left=35mm,right=35mm,top=25mm,bottom=25mm} % Seitenränder/Geometrie
\usepackage{url}
\usepackage{enumitem}
\usepackage{graphicx}
\usepackage{subcaption}
\usepackage{hyperref} % Einsetzen von klickbaren Links
\usepackage{tcolorbox} %Colorbox
\usepackage{tikz} % tikz package zum Erstellen von Vektorgraphiken
\usetikzlibrary{shapes.arrows,decorations.pathreplacing}

\title{Kalibrierungsleitfaden der 5-Hole-Probe}
\author{Alexander Bütow}

\begin{document}

\maketitle

\section{Allgemeines}

Die \textbf{vollständige dynamische Kalibrierung} der 5-Hole-Probe (5HP) im Flug sollte idealerweise unter folgenden Umständen durchgeführt werden:

\begin{itemize}
	\item einmal jährlich (zur Validierung und ggf. Kompensation möglicher altersbedingter Änderungen der Linearität der Drucksensoren)
	\item nach größeren, mechanischen Eingriffen an den Drucksensoren, der 5HP, IMU oder Druckschläuchen (Aus- und Einbau, Sensortausch)
\end{itemize}

Wichtig sind ebenfalls die meteorologischen Voraussetzungen. \textbf{Die Kalibrierung beruht auf den Annahmen eines zeitlich wie lokal stationären Horizontalwindfeldes, sowie des statischen Drucks und eine im Mittel verschwindene Vertikalgeschwindigkeit des Windes.} Daraus ergibt sich, dass das Kalibrierungsprogramm z.B. an der unmittelbaren Vorderseite einer sich nähernden Kaltfront bzw. Tiefdruckgebietes, inmitten einer synoptisch- oder mesoskaligen Konvergenzlinie, sowie eines kräftigen Hochs, möglichst nicht ausgeführt werden sollte.\\

\textbf{Meteorologische Voraussetzungen:}

\begin{itemize}
	\item stabiles Wetter (hinsichtlich Drucktendenz, Windrichtung und -geschwindigkeit)
	\item möglichst geringe vertikale Windgeschwindigkeit auf synoptischer Skala
	\item Flug über planetarischer Grenzschicht möglich
\end{itemize}


\begin{tcolorbox}
	Darüber hinaus sind \textbf{vor} und (optional) nach jedem Flug die Offsetwerte der Drucksensoren zu bestimmen. Das Verfahren wird im Folgenden erläutert und ist mit dieser Box gekennzeichnet.
	
\end{tcolorbox}

\begin{tcolorbox}

	\section{Vor und nach jedem Flug: Offsetwerte für Drucksensoren bestimmen}
	
	\textbf{Vor dem Flug:}
	
	\begin{itemize}
		\item Sicherstellen, dass rote Schutzkappe auf der 5HP montiert ist
		\item Zuschalten einer externen Spannungsquelle ans Flugzeug
		\item Spannungsversorgung des Wingpods zuschalten
		\item Logger für Drucksensoren starten (starten normalerweise automatisch nach anliegender Versorgungsspannung des Wingpods)
		\item 15 bis 20 min als Warmup-Zeit vergehen lassen
	\end{itemize}
	
	Die Druckmesswerte driften in ihre Nulllage; diese können im Postprocessing abgelesen werden.\\
	
	\textbf{(Nach dem Flug (Motor ist abgeschaltet)):}
	
	\begin{itemize}
		\item rote Schutzkappe aufsetzen
		\item Spannungsversorgung wiederherstellen (siehe oben)
		\item 5 min vergehen lassen
	\end{itemize}
	
	Erneut können die Messwerte im Postprocessing abgelesen werden. Die Werte nach dem Flug dienen zur Validierung.

\end{tcolorbox}

\newpage

\section{Bestimmung des statischen Druckdefekts}

Bei diesem Manöver wird die Fluggeschwindigkeit im Horizontalflug variiert.

\begin{figure}[h!]
	\centering
		\begin{tikzpicture}[scale=1, baseline=(current bounding box.north)]
		\draw[->,thick] (0,0) -- (0.5,0);
		\draw[thick] (0.5,0) -- (1,0);
		\node (pic) at (1,0.05) {\includegraphics[scale=0.07,angle=20]{./docmedia/airplane_side.png}};
		\node[below] at (pic) {\tiny $v_{ias} \gtrsim v_{Stall}$};
		\draw[thick] (1,0) -- (3.5,0);
		\node (pic2) at (3.5,0.05) {\includegraphics[scale=0.07,angle=-5]{./docmedia/airplane_side.png}};
		\node[below] at (pic2) {\tiny $v_{ias} \lesssim v_{NE}$};
		\draw[->,thick] (3.5,0) -- (4,0);
		\end{tikzpicture}
		\caption{Variation der angezeigten Fluggeschwindigkeit $v_{ias}$ (Seitenansicht).}
\end{figure}

\begin{itemize}
	\item über die turbulente Grenzschicht aufsteigen bzw. sicherstellen, dass keine Turbulenz spürbar ist
	\item Horizontalflug einnehmen
	\item Flughöhe fortwährend konstant halten
	\item angezeigte Fluggeschwindigkeit $v_{ias}$ mit Hilfe der Motorleistung langsam zwischen Strömungs\-abrissgeschwindigkeit $v_{Stall}$ und Höchstgeschwindigkeit\footnote{Höchstgeschwindigkeit, die mit Motorbetrieb im Horizontalflug erreichbar ist, ansonsten nahe $v_{NE}$.} variieren
	\item Wiederholungen: 5 bis 10
\end{itemize}

\section{Kalibrierung des Anstellwinkels}

Bei diesem Manöver geht es darum, den Anstellwinkel\footnote{Der Anstellwinkel bezieht sich auf die Anströmung, nicht auf den Horizont ($\theta$, \textit{Attitude}).} ($\alpha$, \textit{Pitch}) dynamisch durch wechselnde Flächenbelastungen zu alternieren.

\begin{figure}[h!]
			\centering
			\begin{tikzpicture}[scale=1, baseline=(current bounding box.north)]
			\draw[->,thick] (0,0) sin(1,.2);
			\draw[->,thick] (1,.2) cos(2,0) sin(3,-.2) cos(4,0) sin(5,.2) cos(6,0) sin(7,-.2) cos(8,0);
			\node (pic) at (2,0) {\includegraphics[scale=0.07,angle=-10]{./docmedia/airplane_side.png}};
			\node (pic2) at (5,.25) {\includegraphics[scale=0.07,angle=5]{./docmedia/airplane_side.png}};
			\draw [decorate,decoration={brace,amplitude=5pt,raise=5pt,mirror},yshift=0pt](1,-0.5) -- (5,-0.5) node [black,midway,yshift=-0.5cm] {\tiny$\approx 5$ s};
			\end{tikzpicture}
			\caption{Oszillation des Anstellwinkels $\alpha$ (Seitenansicht).}
\end{figure}

\begin{itemize}
	\item über die turbulente Grenzschicht aufsteigen bzw. sicherstellen, dass keine Turbulenz spürbar ist
	\item kurz Horizontalflug einnehmen
	\item Ausführen möglichst sinusförmiger Oszillationen um die Querachse ($\pm 10^\circ$), Periodendauer ca. 5 s oder kürzer
	\item Wiederholungen: 10 bis 20
\end{itemize}

\newpage

\section{Kalibrierung des Gierwinkels}

Ähnlich wie im voherigen Manöver, stattdessen wird der Gierwinkel alterniert.

\begin{figure}[h!]
	\centering
	\begin{tikzpicture}[scale=1, baseline=(current bounding box.north)]
	\draw[->,thick] (0,0) -- (1,0);
	\draw[->,thick] (1,0) -- (6,0);
	\node (pic) at (1.5,0) {\includegraphics[scale=0.07,angle=-75]{./docmedia/airplane_top.png}};
	\node (pic2) at (3,0) {\includegraphics[scale=0.07,angle=-105]{./docmedia/airplane_top.png}};
	\node (pic3) at (4.5,0) {\includegraphics[scale=0.07,angle=-75]{./docmedia/airplane_top.png}};
	\draw [decorate,decoration={brace,amplitude=5pt,raise=5pt,mirror},yshift=0pt](1.5,-.3) -- (4.5,-.3) node [black,midway,yshift=-0.5cm] {\tiny$\approx 5$ s};
	\end{tikzpicture}
	\caption{Oszillation des Gierwinkels $\beta$ (Draufsicht).}
\end{figure}

\begin{itemize}
	\item über die turbulente Grenzschicht aufsteigen bzw. sicherstellen, dass keine Turbulenz spürbar ist
	\item kurz Horizontalflug einnehmen
	\item Ausführen möglichst sinusförmiger Oszillationen um die Gierachse ($\pm 10^\circ$), Periodendauer ca. 5 s
	\item Flügel horizontal halten (Quer- und Seitenruder kreuzen)
	\item Wiederholungen: 10 bis 20
\end{itemize}

\section{Bestimmung der Offsetwinkel $\alpha_0$ und $\beta_0$ und Validierung des statischen Druckdefekts}

Bei diesem Manöver wird ein \textit{Heading} in wechselnder Richtung (\textit{Reverse-Heading-Maneuver}) oder alternativ ein Quadrats (parallel zu Haupthimmelsrichtungen) abgeflogen.

\begin{figure}[h!]
	\centering
		\begin{subfigure}{0.25\textwidth}
			\begin{tikzpicture}[scale=0.5, baseline=(current bounding box.north)]
			\tikzstyle{every node}=[font=\scriptsize]
			\node[rotate=-60] (airplane) at (0,0) {\includegraphics[scale=0.07]{./docmedia/airplane_top.png}};
			\draw[->,thick, rotate=-60] (0,0) -- (0,5);
			\draw[->,thick, rotate=-60] (0,5) arc (0:45:1.6) coordinate (leg);
			\draw[thick,rotate=-60] (leg) arc (45:-180:-0.3) -- (0,5);
			\draw[->,thick, rotate=-60] (0,5) -- (0,1);
			\draw[->,thick, rotate=-60] (0,0) arc (0:45:-1.6) coordinate (leg2);
			\draw[thick,rotate=-60] (leg2) arc (45:-180:0.3) -- (0,0);
			
			\draw[->, >=latex, blue!40, line width=7pt, rotate=-150] (-3,3) -- node [black,rotate=-60] {Wind} (-3,0.2);
			\draw [decorate,decoration={brace,amplitude=5pt,raise=5pt},yshift=10pt,rotate=-60] (0,0) -- (0,5) node [black,midway,xshift=-10pt,yshift=10pt,rotate=30] {\tiny$\approx 30 - 60$ s};
			\end{tikzpicture}
			\caption{}
		\end{subfigure}
		\hspace{2cm}
		\begin{subfigure}{0.25\textwidth}
			\begin{tikzpicture}[scale=1, baseline=(current bounding box.north)]
			\tikzstyle{every node}=[font=\scriptsize]
			\node (start) at (0,0) {\includegraphics[scale=0.07]{./docmedia/airplane_top.png}};
			\draw[thick] (start) -- (0,2) coordinate (leg1_e);
			\draw[thick] (leg1_e) arc(0:90:0.5) coordinate (leg2_b);	
			\draw[->,thick] (leg2_b) -- (-2.5,2.5) coordinate (leg2_e);
			\draw[thick] (leg2_e) arc(90:180:0.5) coordinate (leg3_a);
			\draw[thick] (leg3_a) -- (-3,0) coordinate (leg3_e);
			\draw[thick] (leg3_e) arc(180:270:0.5) coordinate (leg4_a);
			\draw[->,thick] (leg4_a) -- (-0.5,-0.5) coordinate (leg4_e);
			\draw[thick] (leg4_e) arc(270:360:0.5);
			
			%\draw[->, >=latex, blue!40, line width=7pt] (-1.5,2) -- node [black,rotate=90] {Wind} (-1.5,0);
			%\draw [decorate,decoration={brace,amplitude=5pt,mirror,raise=5pt},yshift=0pt](-2.5,-0.5) -- (-0.5,-0.5) node [black,midway,yshift=-0.5cm] {\tiny$\approx 30 - 60$ s};
			\end{tikzpicture}
			\caption{}
		\end{subfigure}
	\caption{Abwechselndes Fliegen eines Headings (a) oder eines Vierecks (b).}
\end{figure}

\begin{itemize}
	\item über die turbulente Grenzschicht aufsteigen (Alternativ: tiefer Flug über Grund, sofern Grundschichtkonvektion noch nicht stark entwickelt ist)
	\item Höhe und Geschwindigkeit fortwährend halten
	\item \textit{Heading} parallel zur Windrichtung (Alternativ: 0$^\circ$, 90$^\circ$, 180$^\circ$ oder 270$^\circ$) ausrichten
	\item \textit{Heading} 30 bis 60 s halten
	\item 180$^{\circ}$-Wende durchführen bzw. $90^\circ$ beim Abfliegen eines Vierecks
	\item neues \textit{Heading} für die gleiche gewählte Zeit des voherigen Abschnitts halten
	\item 180$^{\circ}$-Wende durchführen bzw. $90^\circ$
	\item Wiederholungen: 5 bis 10
\end{itemize}

Am besten werden die Manöver nicht an ortsgebunden Wendepunkten festgemacht, d.h. der horizontale Windversatz des Flugzeugs während der Durchführung ist sogar \textbf{erwünscht}, da es sich so immer in der gleichen Luftmasse bewegt. Es reicht also aus, nach Uhr und Kompass zu fliegen.

\end{document}